\documentclass[sigconf]{acmart}
\usepackage[utf8]{inputenc}

\title{Security through Inefficiency}
\subtitle{Leveraging Social Friction to Combat Online Manipulation}
\author{Joel Miller}
\affiliation{%
  \institution{University of Illinois at Chicago}
}
\email{jmill54@uic.edu}

\author{Chris Kanich}
\affiliation{%
  \institution{University of Illinois at Chicago}
}
\email{ckanich@uic.edu}

\date{May 2020}

\begin{document}

\maketitle

\section{Justification Statement}

% % % The justification statement briefly explains why the submission is appropriate for NSPW and the chosen submission category. 

Security through Inefficiency is a \textbf{regular submission} for NSPW 2020. In the paper, we reconsider the supremacy of technical efficiency in the design and implementation of Internet communications platforms. We present a new concept, \textit{social friction}, and explore its implications for enabling effective socio-technical security of online communities. Since this paper critiques existing approaches to security and presents a new approach, we feel that it is  appropriate for submission to NSPW in the ``regular submission'' category.

\section{Participation Statement}

% % % The participation statement must specify which author(s) will attend upon acceptance/invitation, that all authors will engage in good faith with the feedback given in the review and revision periods, and that all authors will abide by the NSPW code of conduct.
Should the paper be accepted, Joel Miller will attend the workshop. Chris Kanich would be interested in attending as well, if space allows. We are committed to engaging in good faith with any feedback we receive during every step of the process, and will abide by the NSPW code of conduct.

\thispagestyle{empty}

\end{document}

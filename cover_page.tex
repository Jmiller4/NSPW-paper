\documentclass{article}
\usepackage[utf8]{inputenc}

\title{Social Friction in Online Communities}
\author{Joel Miller and Chris Kanich\\University of Illinois at Chicago}
\date{May 2020}

\begin{document}

\maketitle

\section{Justification Statement}

% % % The justification statement briefly explains why the submission is appropriate for NSPW and the chosen submission category. 

Social Friction in Online Communities is a \textbf{regular submission} for NSPW 2020. In the paper, we question current dogma regarding the fundamental qualities of the internet and analyze the ways those attitudes have affected security research and the design of online communities. We present a new concept, \textit{social friction}, and explore its implications for the design of online communities. Given that we critique existing approaches to security and present a new approach in this paper, we feel that it is  appropriate for submission to NSPW in the ``regular submission'' category.

\section{Participation Statement}

% % % The participation statement must specify which author(s) will attend upon acceptance/invitation, that all authors will engage in good faith with the feedback given in the review and revision periods, and that all authors will abide by the NSPW code of conduct.

We are committed to engaging in good faith with any feedback we receive during any step of the process, and will abide by the NSPW code of conduct in all relevant interactions. 

\end{document}

\documentclass[sigconf,authordraft]{acmart}
\usepackage{tabularx}

% \documentclass[sigconf,authordraft]{acmart} gives same document but with line numbers and "unpublished draft" watermark across whole page

%%%% Proceedings format for SIGPLAN conferences 
% \documentclass[sigplan, anonymous, authordraft]{acmart}

%%%% Proceedings format for conferences using one-column small layout
% \documentclass[acmsmall,authordraft]{acmart}

%% Rights management information.  This information is sent to you
%% when you complete the rights form.  These commands have SAMPLE
%% values in them; it is your responsibility as an author to replace
%% the commands and values with those provided to you when you
%% complete the rights form.
\setcopyright{acmcopyright}
\copyrightyear{2020}
\acmYear{2020}
\acmDOI{10.1145/1122445.1122456}

%% These commands are for a PROCEEDINGS abstract or paper.
\acmConference[NSPW 2020]{New Security Paradigms Workshop 2020}{26-29 October 2020}{North Conway, NH}
\acmBooktitle{New Security Paradigms Workshop '20, October 26--29, 2020, North Conway, NH}
\acmPrice{0.00}
\acmISBN{978-1-4503-XXXX-X/18/06}


%%
%% Submission ID.
%% Use this when submitting an article to a sponsored event. You'll
%% receive a unique submission ID from the organizers
%% of the event, and this ID should be used as the parameter to this command.
%%\acmSubmissionID{123-A56-BU3}

%%
%% The majority of ACM publications use numbered citations and
%% references.  The command \citestyle{authoryear} switches to the
%% "author year" style.
%%
%% If you are preparing content for an event
%% sponsored by ACM SIGGRAPH, you must use the "author year" style of
%% citations and references.
%% Uncommenting the next command will enable that style.
%%\citestyle{acmauthoryear}

\begin{document}


\title{Social Friction and the Distribution of Responsibilities in Online Communities}

%

\author{Joel Miller}
\affiliation{%
  \institution{UIC}
}
\email{jmill54@uic.edu}

\author{Chris Kanich}
\affiliation{%
  \institution{UIC}
}
\email{ckanich@uic.edu}






% % % Abstract

\begin{abstract}
Abstract goes here.
\end{abstract}

%%
%% The code below is generated by the tool at http://dl.acm.org/ccs.cfm.
%% 

\begin{CCSXML}
<ccs2012>
   <concept>
       <concept_id>10002978.10003029</concept_id>
       <concept_desc>Security and privacy~Human and societal aspects of security and privacy</concept_desc>
       <concept_significance>500</concept_significance>
       </concept>
 </ccs2012>
\end{CCSXML}

\ccsdesc[500]{Security and privacy~Human and societal aspects of security and privacy}

% end of copied code from http://dl.acm.org/ccs.cfm.

%%
%% Keywords. The author(s) should pick words that accurately describe
%% the work being presented. Separate the keywords with commas.
\keywords{social networks, sybil attacks}

%%
%% This command processes the author and affiliation and title
%% information and builds the first part of the formatted document.
\maketitle

\section{Introduction}

Introduction goes here.

\section{Social Friction}

\textit{(NOTE: this section is maybe a bit long. My main goal at the start was to give social friction a grounding in sociology/anthropology that gave it more credibility. I haven't been able to find any literature that exactly mapped to the concept. I suppose that this section seeks to justify the introduction of ``social friction'' as a new term instead, while still showing its connections to the humanities)}

We use the term {\itshape social friction} to describe the natural barriers to entry an individual may face when joining a community. If the community is a group of friends, a newcomer seeking to join the friend group will need to gain the trust of the group before they are accepted. If the community is a town, a newcomer must buy or rent property in that town to be considered a resident. If the community is a family, then a newcomer must marry a family member (or be born of a family member) to be considered a member of that family. In each of these scenarios, a newcomer to the community needs to meet certain criteria (accruing trust, buying property, courtship and marriage) before they can attain community membership. We use the term {\itshape social friction} to describe the challenges encountered in meeting those criteria. 

Social friction is related to the concepts of \textit{social barriers} and \textit{rites of passage} from Anthropology and Sociology. {\itshape Social barriers} are used broadly in the literature to describe non-technological hindrances that make it more difficult for an individual or group to complete an action, whether that action be joining a social movement \cite{klandermans1987potentials}, visiting a city park \cite{cutts2009city}, accepting a new energy source\cite{pasqualetti2011social}, or achieving economic empowerment\cite{woolcock2000removing}. The varied contexts in which the idea of a social barrier appears highlight a key difference between that term and our idea of social friction: that of scope. A social barrier can refer to a broad set of phenomena, whereas social friction is more specific. In fact, one could think of social friction as the specific social barriers that hinder an individual from joining a community (rather than, say, accepting a new energy source). 

We also seek to distinguish the idea of social friction from the idea of social barriers in an attitudinal sense \textit{(NOTE: there has got to be a better word than ``attitudinal'' to use here...)}. In the papers we encountered during our literature review, a social barrier was always framed by the authors as something negative that should be torn down if possible, and with good reason -- many of these social barriers did indeed stand in the way of positive changes. But we do not see social friction as an inherently negative phenomenon. In fact, social friction is beneficial to communities in so far as it provides a natural vetting process against potentially malicious outsiders. Imagine, for example, if any stranger could walk into your house and announce themselves as a new member of your family -- you may have some doubts about that stranger's intentions. Social friction (manifesting here as the requirement that this stranger gain the trust and love of an unmarried family member before joining the family) is what keeps such a thing from happening. 

The idea of social friction also bears similarity to the idea of a rite of passage, which emerged from Anthropological research and has since seen use in other disciplines \cite{blackwell_ROP,guha2011routledge}. However, if social barriers are too broad of a concept to mesh well with our idea of social friction, rites of passage are instead too specific. A rite of passage denotes a well-defined ritual which occurs at a specific point in time, whereas the actions needed to overcome social friction (e.g. gaining trust) are not necessarily demarcated by any one event and can occur gradually over time. Nevertheless, some of the actions necessary to overcome social friction could be seen as rites of passage (e.g. a meeting to finalize the purchase of a house), so there is some overlap between the two concepts. 

One can see social friction as a mechanism through which communities can provide security for themselves -- a tool wielded by community members to make sure that only people they trust join the community. Furthermore, one can notice that the amount of social friction needed to join a community is often correlated with the amount of harm that an unwanted community member could cause: joining a house party is a process with relatively little friction, whereas joining a family is a high-friction process. If we consider social friction as a source of security for communities, this correlation makes perfect sense: it is natural to put more effort into protecting something that is more valuable.

\section{Friction on the internet?}

How, then, should we think of friction with regards to online communities? The common consensus is that many social networks are relatively low-friction environments, and in general the internet has been praised for allowing ideas and artifacts to travel between communities with relatively little hindrance. In general, openness is a defining characteristic of the internet \cite{bechmann2014ubiquitous, lessig2002future}, and a recent survey of Sybil attack defense and detection mechanisms cites openness as a fundamental property of online social networks \cite{al2017sybil}.

Our intention is not to argue that the fundamental openness of the internet needs to be changed, but rather that designers of online social networks do should not necessarily accept the requirement of openness as a given. Within the domain of social networks, the openness prized by so many has led to botnets that can carry out disinformation campaigns and generally wreak havoc \cite{boshmaf2013design, kramer2014experimental, wu2013detecting, messias2013you, benkler2018network, bessi2016social}. 

At the end of the previous section, we noted an anecdotal correlation between the amount of social friction around a real-world community and relative harm a malicious outsider could do to that community, were they able to gain membership. It would appear that many online communities do not fit with in with this trend -- they have low social, friction, but unwanted actors (i.e. bots) who enter them can cause a lot of harm. Accordingly, we propose that introducing friction to social networks, specifically online deliberation platforms, could be an effective way to combat bad-faith behavior.

Much of the harm and disinformation in online social networks is caused not by real people but by bots (i.e. Sybil attacks). In line with the prioritization of the internet's openness, most approaches to Sybil attacks have been reactive in nature \cite{al2017sybil,Wei2012sybildefender,boshmaf2015integro,cao2012aiding,cao2013sybilfence,shi2013sybilshield, danezis2009sybilinfer,yang2014uncovering,wang2012social,varol2017online, lingam2018detection}\textit{(NOTE: not sure this is the right approach -- citing many papers that use one method doesn't prove anything about which method is used more than another in the literature)} -- that is to say, the authors seek to detect and remove Sybil identities that have already made their way into the network, instead of taking an approach that limits who can join (although some research has been along these lines \cite{yu2006sybilguard,yu2008sybillimit,Tran2011Gatekeeper}). Foundational research on Sybil attacks suggests that a reactive approach (i.e. letting arbitrary nodes join the network and trying to detect which ones are Sybil nodes) is much less effective compared to approaches that gate who can enter the community, except under exceedingly rare circumstances \cite{douceur2002sybil}. 

Furthermore, the presence of bots is not the  only factor contributing to the political disarray of most social networks: real humans can also take up the task of spreading disinformation and propaganda, sometimes for pay \cite{bradshaw2017troops}. Also problematic are the data collection practices of firms that use personal data to create targeted political messaging. The public's negative response to such practices are perhaps best exemplified in the Facebook-Cambridge Analytica scandal \cite{CambridgeAnalyticaNYT}.

Overall, we worry that in the context of social media, and more specifically in the context of social media discussions related to politics, the openness of the internet has allowed for more harm than good. Specifically, we feel that the idea that any user can receive incoming information from any other source has been abused by malicious parties who inundate regular users with so much bad-faith information that distinguishing between the good from the bad, real people and credible sources from bots and disinformation, is all but impossible.  


\textit{NOTE: I also wonder if some of this information would be better suited for the introduction.}

\section{Mastodon and distributed administration/hosting}

Mastodon is microblogging service similar to Twitter. Users share short messages called ``toots'', which they can favorite, reply to, and ``boost'' (an analog to Twitter's retweet functionality). Mastodon does not have the recommendation features that Twitter has (i.e. ``who to follow'' suggestions), but recent work has shown that implementing recommender systems on top of Mastodon is possible \cite{trienes2018recommending}. Like Twitter, Mastodon is primarily developed by a centralized team. However, Mastodon's software is open source.

The difference between Mastodon and Twitter (or any other social network) is the radical nature in which its communities and administrative privileges are distributed. Mastodon is made up of many interconnected servers, each hosted by an individual or party who need not be connected to the developers of Mastodon, and when a user joins Mastodon they choose a specific server to join (although they can migrate their content to accounts on other servers later if they so choose). In general, there is nothing stopping a user on one server from interacting with a user on another server {\itshape a priori}. Moreover, the developers of Mastodon hold none of the administrative privileges. Instead, the parties who run each individual mastodon server has administrative priveleges over what goes on on that server, including who is allowed to join and who is allowed to see posts on the server. In this way, Mastodon is decentralized with respect to its distribution of administrative privileges and its distribution of server-hosting responsibility.

This naturally allows for a framework by which communities can create social friction around themselves. Specifically, each server is free to set their own guidelines on who is allowed to enter. These guidelines can be social (e.g. community members discuss whether or not they want to let the outsider in), technological (e.g. a user is let into the server for that residents of a town use to talk local politics if they can provide a crpytographic proof of their residence in the town), or a mix of both. 

Letting communities set their own security and membership guidelines has another advantage: it reduces the scalability of attacks. Any bot-detection algorithm implemented by an open social network like twitter will have the drawback that an any successful evasion of the algorithm will scale quite well. That is to say, since all communities on twitter are uniform with regards to their protection under a bot-detection algorithm, a strategy for evading the algorithm in one community will also work in any other community. In contrast, letting every decentralized server create their own membership requirements is a security strategy that will not lend itself well to wide-scale attacks. Since each server is encouraged to customize their membership requirements, an attack that works well on one server is by no means guaranteed to work on any other server. This is an important advantage because it greatly reduces the economic efficiency of any potential attack. Moreover, since under our model the development privileges are still centralized, the developers of a system like Mastodon could still create a bot-detection algorithm and let servers adopt it for an extra layer of security.

We do not mean to suggest that Mastodon in particular needs to become the center of research or development efforts centered around social friction -- only that its operational model has many potentially positive qualities with regard to enabling high-quality deliberation.

\section{overlaying friction on other social networks}

The idea of social friction can also be used to build overlays on other social networks. For example, one could imagine a tool that lets Twitter users define their own metric of trust relative to another user (e.g. number of followers in common, distance away in the follow graph, etc) and augments the user's twitter feed by only showing the user tweets from other users who attain a threshold trust score, under whatever metric the user defines. Since each user is free to customize their own metric of trust, we gain an advantage from decentralization similar to the advantage gained by letting each individual Mastodon server set its own membership rules. That is to say, for users Alice and Bob, a strategy that lets an attacker many bots that can unfairly gain Alice's trust will not necessarily succeed in creating bots that can unfairly gain Bob's trust.

But we also note that it might not be appropriate to incorporate the idea of social friction into every online network. Specifically, we see social friction as being extremely useful to the operation of online deliberation \cite{semaan2015designing} and participatory budgeting platforms \cite{de1998participatory,shah2007participatory,wampler2010participatory,cabannes2004participatory}, via which participants can debate each other and vote on policy decisions or recommendations. Importantly, in the context of political discussion, we do not necessarily wish to create communities that are divided along political lines, but communities free of bad-faith outsiders who wish to sway the conversation in one direction or another. 

A fitting analog might be the idea of Deliberative Democracy first proposed by Fiskin \cite{fishkin1991democracy}, which has manifested most recently in the America in One Room project \cite{AmericaInOneRoom}. This project saw Americans from all walks of life invited to a central location in Texas where they debated topics germane to American politics topics with each other. The organizers of this event did not seek to exclusively invite participants with a specific political orientation, but they {\itshape did} seek to only invite participants who were from America, given that the event was centered around debating American politics. In the same way, our hope is that online deliberation platforms can use social friction to keep out bad-faith outsiders while still allowing for healthy political debate.

\section{a taxonomy of social networks}

In order to better understand the axes along which various social networks are centralized or decentralized, we have taxonomized several popular social networks  (Table \ref{tab:taxonomy}).

%%% Link to taxonomy in my google drive : https://docs.google.com/spreadsheets/d/1Q9rg_i6YAEYkcFf2JBfe5wJIKaecT4_MpXmSPnUIFSE/edit#gid=0

\begin{table*}
  \caption{A taxonomy of online networks}
  \label{tab:taxonomy}
  \begin{tabularx}{\textwidth}{llXX}
    \toprule
    Platform & Centralization of source code & Centralization of administrative privileges & Centralization of community hosting \\
    \midrule
    Twitter  & Centralized & Centralized\\
    Facebook & Centralized & Semi-centralized: moderators of Facebook groups have some administrative privileges & Centralized\\
    Reddit   & Centralized & Decentralized & Centralized  \\
    Slack/ discord & Centralized & Decentralized & Centralized \\
    Group Chats & Centralized & N/A -- no one has administrative privileges & Centralized \\
    Mastodon & Decentralized (open source) & Decentralized & Decentralized\\
    Blockchain Apps & Decentralized (open source) & N/A -- application-dependent & Decentralized\\
    \bottomrule
  \end{tabularx}
\end{table*}

Of these networks, three have decentralized administrative privileges (Reddit, Slack/Discord, and Mastodon), and of those three only Mastodon is decentralized with regards to community hosting. But without decentralized community hosting, administrative privileges cannot be truly decentralized, since the power given to administrators can still be altered by the central entity who hosts all the communities. For example, there is nothing stopping Reddit from changing its policy on subreddit administration and transferring some moderator power to a centralized in-house team.

We can also see the ``report'' functions on Twitter and Facebook (via which anyone can flag an inappropriate post for moderator review) as an attempt to distribute some administrative capability throughout the network. However, any reports made under such a scheme must still pass through a centralized bottleneck of in-house moderators. 

\section{Future Work}

We hypothesize that the centralization or decentralization of a network's administrative privileges and hosting ability has a profound impact on the way information flows in that network and the amount of social friction communities can are able create around themselves. Moreover, we hypothesize that allowing communities to establish social friction around themselves.

Our future work will have two main goals: to validate our hypothesis that increased social friction can benefit communities involved in political discussions, and to design technological mechanisms that will enable communities to create social friction around themselves in online spaces. 

To test our hypothesis on the effectiveness of social friction, we will conduct studies that measure the quality of conversations across online communities that exhibit various amounts of social friction. One promising set of communities to examine are Reddit's ``subreddits'', many online communities each with their own sets of rules.

On the technological side, we have begun to design a privacy-preserving cryptographic protocol which uses zero-knowledge proofs \cite{goldreich1991proofs} to allow agents to show that certain facts about their average location over time. Such a system could be used to let users create online communities for discussions germane to a certain geographic area where only the residents of that geographic area can participate, while still respecting the privacy of all participants.

\section{Conclusion}

We have presented a new viewpoint on security that seeks to question the de-facto acceptance of the internet's openness. We suggest that in certain scenarios, the openness of the internet has the potential to do more harm than good, and that this argument is supported by recent events in American politics. To further support that hypothesis, we define the notion of social friction and consider the ways in which it can be seen as a tool that communities can use to provide security for themselves. We examine the structures of different social networks, creating a taxonomy along which different networks are classified according to how they distribute development, administrative, and hosting responsibilities. We hypothesize that networks with decentralized administrative and hosting responsibilities are more likely to support communities guarded by social friction, and hypothesize that such communities will be more conducive to high quality discussion. Our goal is to validate those hypothesis via experimental evidence, and to design technological mechanisms that will enable social friction.

\begin{acks}
The authors would like to thank Emma Hanlon and Gabriel Eisen for helpful discussions about Sociology and Anthropology.
\end{acks}

\bibliographystyle{ACM-Reference-Format}
\bibliography{bibliography}


\end{document}
\endinput